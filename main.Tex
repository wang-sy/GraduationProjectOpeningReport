\documentclass{article}
\usepackage[UTF8]{ctex}
\usepackage[breaklinks,colorlinks,linkcolor=black,citecolor=black,urlcolor=black]{hyperref}
\usepackage{graphicx}

%% 设置封面格式
\title{毕业设计(论文)开题报告\protect\\集群管理与调度系统研究与实现}
\author{王赛宇 20184199}
\begin{document}
\maketitle

% section 1, 课题任务及意义
\section{课题任务及意义}

% subsection 1.1, 课题任务
\subsection{课题任务}
  集群管理与调度系统是一种微服务场景下的服务发布、运维系统。本课
题要求借助已有工具设计并实现一用于支持校内学生进行微服务开发的集
群管理与调度系统,具体完成的任务包括:
    (1)查阅资料学习现有的集群管理与调度系统;
    (2)针对校内微服务开发场景,设计系统功能;
    (3)使用Golang,借助已有工具实现微服务的发布、运维等功能;
    (4)使用Vue,提供友好的操作页面,供开发人员使用;
  本课题要求阅读大量的文献资料,熟悉实际开发、上线流程,掌握Golang、
Kubernetes\cite{kubernetes}、Helm、Docker\cite{dockerContainer}、
Devops\cite{devopsForDeveloper}等知识与工具并在实践中应用。

% subsection 1.2, 课题研究意义
\subsection{课题研究意义}

\subsubsection{传统软件开发运维模式及其局限性}
  随着互联网的高速发展,各企业规模日益扩张,所涉及的应用领域也愈
发广泛。按照以往的方式,企业需要雇佣专业的运维人员来对服务器上的环
境进行维护和管理。随着业务的扩张,需要投入大量人力以确保应用正常运
行,还需要人为的对服务器的io、内存、cpu资源进行划分,从而减少资源
浪费现象。因此企业应用部署困难、资源浪费的的问题亟待解决。\cite{cloudExtend}

  单体软件开发方法是以往的行业惯例。开发人员管理着庞大的代码仓
库,任何编码更改都会影响整个系统,因此必须仔细协调和测试这些更改\cite{cloudNativeVSMonolithicApplications},
这导致软件的发布周期较长。在发布版本时,单体软件必须重新部署整
个应用程序。单体软件难以根据单个功能模块的访问量来确定占用资源大
小,面对个别接口访问量突增时只能通过增加机器数量保证所有服务可用。

\subsubsection{微服务开发模式及其解决的问题}
  微服务架构风格是一种将单个应用程序开发为一组小服务的方法,每个
小服务都在自己的进程中运行,并通过轻量级机制(通常是 HTTP 协议 
API)进行通信\cite{microservicesSummary}。在微服务架构中,不同的服务之间互不耦合,可以单独修改、
发布。

  容器是完整运行时环境的虚拟实例,它承载程序或微服务并允许其独
立于底层基础设施运行。通过这种方法,可以轻松地分离硬件和软件组件,
保证应用程序跨不同系统的完全可移植性。在使用容器的微服务架构团队
中,开发人员可以专注于业务逻辑和应用程序依赖关系,而运维团队可以专
注于部署和管理。

  微服务架构通过将大型单体服务拆分为多个小服务分别维护的方法,解
决了单体软件中仓库巨大、变更“牵一发而动全身”的问题,在某一服务的
访问量激增时,运维人员可以仅针对某一服务进行扩容;容器的引入使得运
维人员在进行实际部署时,不再考虑同一台机器上复杂的依赖关系。根据
Kong最近的一项调查\cite{kongMicroserviceRepoert},84\% 的美国公司已经将微服务用于应用程序开
发,89\% 的受访 IT 经理认为,向这种新架构模型的过渡使他们在未来具有
竞争优势。

\subsubsection{集群管理与调度系统及其意义}
  在实际生产过程中,运维人员不会通过容器管理工具对服务的每一个
实例进行逐一管理,而是通过集群管理与调度系统(下称系统)进行半自动
化管理,运维人员将服务的描述信息提交到系统,随后系统自动完成服务的
部署、重启、复制、伸缩等功能。系统减少了运维人员的重复劳动,在复杂
集群中混部服务时不再需要人为的对服务器的资源进行划分,增加了资源的
利用率。微服务、容器、devops这三板斧,组成了一个高效、易于维护的
新型开发、部署、运维生态“云原生”。

  但是云原生往往难以落地,问题出现在devops(特别是集群管理与调
度系统)本身的复杂特性上。虽然该类系统提供了看似完美的自动化运维功
能,但同时这类系统学习曲线陡峭,需要学习繁杂的系统概念、提供复杂的
配置文件。在校园乃至小型公司中,很难有能够熟练掌握此类系统的人才。
因此,本课题意在提供一个概念简单、易于上手的轻量级集群管理与调度系
统,以降低大学校园内开发者的上手门槛,降低其入门云原生开发的难度。

% subsection 1.3, 国内外发展现状
\subsection{国内外发展现状}
\subsubsection{敏捷开发:极限编程与gitflow}
  传统的瀑布模型无法解决互联网应用中模糊的、变动性强的业务需
求。《敏捷软件开发宣言》\cite{manifestoForAgile}于2001年提出,这是一种周期性的、以人
为核心的、循序渐进的开发过程模型,它假设客户对产品需求的认知是模糊
的,提倡先编写软件原型,再根据实际场景对软件进行修正并再次发布版
本。极限编程(Extreme Programming)\cite{extremeProgramming}是一套从敏捷开发实践中总结出
的方法论,它指出应当将项目分解为迭代并用迭代机会驱动每个迭代,在开
发过程中需要优先编写测试用例,完成每个迭代后需要及时运行单元测试
并合入主干。

  开发者往往用git\cite{git}工具管理代码,git可以从远程仓库拉取代码进行
开发,并在开发完成后将修改推送回远端仓库。git提供了强大的分支模型,
开发者可以从主分支中fork个人分支独立开发并在开发完成后合入主分支。
gitflow\cite{gitflow}是一种git工作流,它根据敏捷开发流程对分支进行区分:
\begin{itemize}
  \item \textbf{Master} 分支存储正式版本,只从其他分支合入,不允许直接提交;
  \item \textbf{Develop} 分支集成测试最新合入的开发成果;
  \item \textbf{Feature} 分支用于开发人员提交代码并进行自测,自测完成后通过
    Release分支合入Master分支;
  \item \textbf{Release} 分支用于发布版本;
  \item \textbf{Hotfix} 用于修复线上问题;
\end{itemize}

  使用gitflow工作流的团队会频繁的执行Feature分支到Develop分支
的合并操作,代码合入Develop前需要运行单元测试、接口测试,合入
Develop后,需要将最新版本的服务发布到测试环境中进行试运行。在这样
的工作流程下,人工测试、运维方式已经无法满足日益频繁的合入需求。

\subsubsection{Devops}
  敏捷开发落地需要自动化工具支持,DevOps是为了更好地解决研发和
运维之间的关系而提出的。它通过自动化工具完成项目的分布、部署、运维
等操作,使得开发人员可以高频的、流畅的完成代码提交、构建并投入下一
轮迭代。
\begin{figure}[ht]
  \centering
  \includegraphics{figures/devops_flow.jpg}
  \caption{Devops开发循环\cite{devops}}
  \label{fig:label}
\end{figure}

  它将软件的开发流程描述为由八个阶段组成的闭环(如图1所示),环
左侧的任务通过产品人员驱动、开发人员完成,环右侧的任务通过Devops
工具完成。

\subsubsection{Docker与Kubernetes}
  传统的云服务通过硬件虚拟化技术将一台服务器虚拟化为多台虚拟机,
每台虚拟机上都将运行一个操作系统,应用程序直接运行在操作系统上。应
用程序的依赖需要由运维人员手动安装,这导致项目部署困难,拓展性差。

  namespace\cite{linuxNamespace}是Linux内核用来隔离内核资源的方式,它通过分组隔
离的方式让一些进程只能看到与自己相关的一部分资源,通过namespace
机制可以对用户、UTS、IPC、网络等资源进行隔离。Cgroup\cite{linuxCgroup}是linux
内核用来限制进程资源的机制,通过这种机制,可以实现对Linux进程或
进程组的资源限制、隔离和统计功能。
  
dotCloud公司的Docker将应用程序与其依赖打包,在部署时,应用程
序及其依赖以进程的形式运行在宿主机上,底层通过namespace与cgroup
进行隔离与限制。Docker可以降低安装依赖带来的运维成本,同时以进程
为核心的轻量虚拟化技术降低了性能损耗。
  
  k8s(kubernetes)是由Google开源的容器编排工具,它提供了一个可弹
性运行分布式系统的框架,它提供了服务发现和负载均衡、存储编排、自动
部署和回滚、自我修复等重要功能,Docker可以作为k8s的容器运行时系
统。

% section 2, 重点研究内容及技术实现途径
\section{重点内容及技术实现途径}

% section 2.1, 重点内容
\subsection{重点内容}
  本项目致力于实现一个易于使用的项目上线平台,该平台针对开发人员
的项目上线行为进行分析,通过已有工具重点解决集群部署、上线、维护
过程中:
(1)容器编排、资源限制、访问隔离等问题;
(2)项目上线时的配置信息获取问题;
(3)线上环境中,基础设施组件部署问题;
(4)线上问题排查功能;
% section 2.2, 技术实现途径
\subsection{技术实现途径}
  针对上述的四个问题,本节中一一讨论实现途径。

\subsubsection{容器发布、资源限制、访问隔离实现途径}
  用户在开发完成后,需要将自己的项目打包为Docker镜像,再通过本
平台发布。发布过程中,用户需要指定:服务的名称、容器的副本数量、对
外暴露的端口、容器的环境变量。平台在接收到用户的请求后,会创建
Kubernetes中的Deployment\cite{Deployments}资源,用于实现容器的部署与编排。在部署
过程中,平台也会将环境变量注入容器,并对容器占用的内存、CPU、IO等
资源进行限制。

  到此为止,容器部署的基础功能已经实现。平台接收到用户的部署请求
后,会将多个Pod\cite{Pods}部署在集群中。但是Pod之间只能通过Pod在集群中的
VirtualIP进行访问。因此平台会根据用户给出的服务名称,在集群中创建
Kubernetes中的无头服务(Headless Service\cite{Service})资源。Kubernetes在检测到
Headless Service资源后,会将ServiceName到PodIPs的映射关系记录到
CoreDNS中。 相同项目下的其他服务想要访问该服务可以直接通过
ServiceName:prot的形式访问,同时,不同项目下服务无法直接访问,这是
由于每一个项目拥有一个独立的Namespace,CoreDNS在默认情况下无法
跨Namespace解析。

  当需要跨项目访问时,本平台提供两种方案:(1)通过Ingress\cite{Ingress}资源,将
服务暴露到外网访问,其他项目直接通过外网访问;(2)通过ExternalName\cite{ClusterNetWork}
资源,将服务暴露给集群内的所有项目访问,达到跨Namespace访问的目
的;和本平台合作的微服务网关项目会提供第三种方案,即通过网关进行访
问,这种方案的优势在于可以进行更细粒度的访问控制。

\subsubsection{项目上线时的配置信息获取}
  程序在初始化过程中往往需要配置信息的支持,一般Web项目需要连
接数据库,这就需要数据库的用户名、密码、主机名、端口号、数据库名和
其他配置信息。这些信息有一定的私密性,无法直接存储在代码仓库或直接
硬编码在代码中。因此,需要本平台提供动态的配置文件管理功能,平台将
依托Kubernetes的ConfigMap\cite{ConfigMaps}记录用户给出的多个配置文件信息,并且
在容器初始化前通过Path+SubPath直接以文件的形式挂载到用户给出的
路径下。

\subsubsection{线上环境的基础设施组件}
  线上环境往往会依赖数据库、缓存、消息队列等基础设施,这类基础设
施不应该由用户手动部署,而应当由平台直接提供,用户只需要通过ui界
面就可以完成这类服务的部署与管理。

  本课题准备从Mysql数据库入手,通过Helm创建标准化的Chart,用
户在UI界面完成Value的编辑后,直接生成相应资源,提交到集群中即可。
同时,将在集群中部署一Mysql可视化管理工具,解决用户网段不同,无法
远程连接Mysql数据库的问题。

\subsubsection{线上问题排查功能}
  开发人员往往会通过Log输出日志,并在出现问题时,通过线上日志
对问题进行排查。本课题期望借助Kubernetes中的logs功能,捕获相应
Pod中的日志,并返回给用户,来帮助用户排查线上问题。除此之外,本项目期
望在前端中嵌入一控制台,并通过WebSocket连接到后端,通过proxy直接访问
问题容器的shell排查线上问题。


% section 3, 课题预期成果
\section{课题预期成果}

% section 3.1, 课题预期成果
\subsection{课题预期成果}
  本课题预期成果是一个可用性强的服务上线系统,系统包括完整的前
后台服务。用户可以通过前端操作,完成项目的发布、更新、扩容、维护等
操作。
% section 3.2, 课题预期特色
\subsection{课题预期特色}
  微服务化、集群化的开发、管理流程往往在小公司、校园内难以落地。
这是因为当前主流的Kubernetes等工具体系较为繁杂,集群本身的搭建、
管理较为困难,因此本课题期望构建一个完整的项目上线、运维系统,降低
了学生、小公司进行云原生开发的难度。当前计算机专业在校生除去少数的
算法工程师外,大多数在未来都会从事后台开发、前端开发相关工作。与以
往学校内的开发流程相比,本系统可以帮助在校生更快的适应新时代开发
方式。

% section 4, 进度计划
\section{进度计划}
\begin{center}
  \begin{tabular}{|c|c|c|}
    \hline 序号&起止周次&工作内容\\
    \hline 1&1周至  2周& 查找文献资料 \\
    \hline 2&3周至  4周& 学习已有服务发布平台 \\
    \hline 3&5周至  6周& 设计课题服务发布平台 \\
    \hline 4&7周至  10周& 实现后台部分 \\
    \hline 5&11周至 14周& 实现前端部分,撰写论文 \\
    \hline 6&15周至 16周& 完善论文和程序,参加答辩 \\
    \hline
  \end{tabular}
\end{center}

\newpage
% section 5, 参考文献
\section{参考文献}
\begin{thebibliography} {99}
    \bibitem{kubernetes} Burns B , Grant B , Oppenheimer D , et al. Borg, Omega, and Kubernetes[J]. Communications of the ACM, 2016, 59(5):50-57.
    \bibitem{dockerContainer} Bernstein D . Containers and Cloud: From LXC to Docker to Kubernetes[J]. Cloud Computing, IEEE, 2014, 1(3):81-84.
    \bibitem{devopsForDeveloper} M Hüttermann. DevOps for Developers[M]. Apress, 2012.
    \bibitem{cloudExtend} 李渊. 基于云计算的 Web 应用部署与扩容系统[D].华中科技大学,2012:5.
    \bibitem{cloudNativeVSMonolithicApplications} SPARKFABRIK. Microservices and Cloud Native Applications vs. Monolithic Applications[EB/OL].(8 October 2021)[5 December 2021]. https://blog.sparkfabrik.com/en/microservices-and-cloud-native-applications-vs-monolithic-applications
    \bibitem{microservicesSummary} James Lewis , Martin Fowler. Microservices a definition of this new architectural term[EB/OL].(25 March 2014)[5 December 2021]. https://martinfowler.com/articles/microservices.html
    \bibitem{kongMicroserviceRepoert} Kong. 2021 Digital Innovation Benchmark[EB/OL].(2021)[5 December 2021] https://konghq.com/resources/digital-innovation-benchmark-202
    \bibitem{manifestoForAgile}agilemanifesto.org . Manifesto for Agile Software Development[EB/OL].(2001)[11 December 2021] http://agilemanifesto.org/iso/en/manifesto.html
    \bibitem{cicdRedHeat}Red Hat. What is CI/CD?[EB/OL].(31 January 2018)[11 December 2021] https://www.redhat.com/en/topics/devops/what-is-ci-cd
    \bibitem{extremeProgramming}Martin Fowler. The New Methodology[EB/OL].(13 December 2005)[11 December 2021] https://martinfowler.com/articles/newMethodology.html 
    \bibitem{git}git. git[EB/OL].(2021)[11 December 2021] https://git-scm.com/
    \bibitem{gitflow}Vincent Driessen . A successful Git branching model[EB/OL].(05 January  2010)[11 December 2021] https://nvie.com/posts/a-successful-git-branching-model/
    \bibitem{devops}黄敏珍.CMMI、敏捷开发和DevOps在项目管理实践中的应用[J].项目管理技术,2020,18(09):91-95.
    \bibitem{linuxNamespace}Michael Kerrisk. namespaces(7) — Linux manual page[EB/OL].(27 August 2021)[11 December 2021] https://man7.org/linux/man-pages/man7/namespaces.7.html
    \bibitem{linuxCgroup}Michael Kerrisk. cgroups(7) — Linux manual page[EB/OL].(27 August 2021)[11 December 2021] https://man7.org/linux/man-pages/man7/cgroups.7.html
    \bibitem{Deployments}The Kubernetes Authors. Deployments.[EB/OL].(17 September 2021)[25 December 2021] https://kubernetes.io/docs/concepts/workloads/controllers/deployment/
    \bibitem{Pods}The Kubernetes Authors. Pods.[EB/OL].(17 September 2021)[25 December 2021] https://kubernetes.io/docs/concepts/workloads/pods/
    \bibitem{Service}The Kubernetes Authors. Service.[EB/OL].(17 September 2021)[25 December 2021] https://kubernetes.io/docs/concepts/services-networking/service/
    \bibitem{Ingress}The Kubernetes Authors. Ingress.[EB/OL].(17 September 2021)[25 December 2021] https://kubernetes.io/docs/concepts/services-networking/ingress/
    \bibitem{ClusterNetWork}The Kubernetes Authors. Cluster Networking.[EB/OL].(17 September 2021)[25 December 2021] https://kubernetes.io/docs/concepts/cluster-administration/networking/
    \bibitem{ConfigMaps}The Kubernetes Authors. ConfigMaps.[EB/OL].(17 September 2021)[25 December 2021] https://kubernetes.io/docs/concepts/configuration/configmap/
  \end{thebibliography}

\end{document}